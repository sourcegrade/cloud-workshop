\section{Docker-Commands}\label{sec:docker-commands}
\subsection{Nützliche Commands}

\begin{frame}
    \slidehead
    \vspace{-1em}
    \vspace{-2pt}
    \begin{itemize}[<+->]
        \item \bashcommand{docker ps} Zeigt alle laufenden \textbf{Container}
        \item \bashcommand{docker image ls} Zeigt alle lokal verfügbaren \textbf{Images}
        \item \bashcommand{docker volume ls} Zeigt alle \textbf{Volumes}
    \end{itemize}
\end{frame}

\subsection{Compose-Commands}
\begin{frame}
    \slidehead
    \vspace{-1em}
    \vspace{-2pt}
    \begin{itemize}[<+->]
        \item \bashcommand{docker compose up} Startet das Deployment (-d um es im Hintergrund zu starten)
        \item \bashcommand{docker compose down} Stoppt das Deployment
        \item \bashcommand{docker compose logs} Zeigt die Logs von den \textbf{Containern}
    \end{itemize}
\end{frame}


\subsection{Debugging}
\begin{frame}
    \slidehead
    \vspace{-1em}
    \vspace{-3pt}
    \begin{itemize}[<+->]
        \item \bashcommand{docker container exec [OPTIONS] CONTAINER COMMAND [ARG...]} führt einen Command im laufendem Container aus
        \item \bashcommand{docker debug {CONTAINER|IMAGE} } Führt eine Shell im container aus (aktuell in der Beta)
        \item \bashcommand{docker compose logs} Zeigt die Logs von den \textbf{Containern} die aus der \textbf{docker-compose} der aktuellen directory sind
    \end{itemize}
\end{frame}
