\section{Docker-Registry}\label{sec:docker-registry}
\subsection{Was ist eine Registry}

\begin{frame}
    \slidehead
    \Large
    \centering
    \vspace{-2pt}
    \begin{itemize}[<+->]
        \item Eine Docker-Registry speichert verschiedene \textbf{Images} ab.
        \item Docker stellt dazu selbst eine Registry zur verfügung unter \\
              https://hub.docker.com (Standard falls keine Registry spezifiziert)
        \item Jeder kann selbst aber seine eigene Registry hosten ähnlich zu \textbf{Git}
    \end{itemize}
    \normalsize
    \only<4->{
        \bashcommand{docker image pull myreg.local:5000/testing/test-image}
        \bashcommand{docker image push myreg.local:5000/testing/test-image:latest}
    }
\end{frame}

\subsection{Versionierung}

\begin{frame}
    \slidehead
    \Large
    \begin{itemize}[<+->]
        \item Die Images bekommen alle einen \textbf{Tag} zur Versionierung, \\
              wie z.B. \textbf{:latest}.
        \item Dieser wird in den commands und in der docker-compose Datei an das Ende vom Image-Namen gehängt,
              um die spezielle Version auszuwählen.
        \item Dabei steht \textbf{:latest} meistens für die letzte stabile Version.
    \end{itemize}
\end{frame}

\subsection{Auswahl in der Docker-Compose}
\begin{frame}[fragile]
    \slidehead
    \Large
    \centering
    Falls \textbf{Images} aus einer speziellen Registry gepullt werden sollen, muss der Image-Bezeichner
    mit der Registry geprefixt werden.
    \vspace{1em}
    \normalsize
    \begin{codeBlock}{minted language=yaml}
        services:
        image:myreg.local:5000/testing/test-image:latest
    \end{codeBlock}
    \bashcommand{docker run myreg.local:5000/testing/test-image:latest}
\end{frame}
