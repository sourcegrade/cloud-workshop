\section{Kubernetes-Intro}
\subsection{Ich will mehr}

\begin{frame}
    \slidehead
    \Large
    \centering
    Was tun, wenn man mehrere Rechnerknoten automatisiert mit Containern bespielen will,
    damit man das nicht immer alles selbst machen muss. \\
    \vspace{3em}
    Zum Glück gibt es \textbf{Kubernetes}
\end{frame}

\begin{frame}
    \slidehead
    \Large
    \begin{itemize}[<+->]
        \item Kann aus beliebigen Rechnerknoten gebaut werden
        \item Falls korrekt eingerichtet auch redundant
        \item Eingebautes Load-Balancing zwischen nodes
        \item Benutzt wie Docker die gleichen Container
        \item Konzept wie bei Docker, nur auf mehreren Nodes
        \item Braucht andere konzepte, da Container möglichst stateless zur Skalierung sein sollen.
    \end{itemize}
\end{frame}

\begin{frame}
    \centering
    \includesvg[height = 19em]{../pictures/components-of-kubernetes-smaller.svg}
\end{frame}
\begin{frame}
    \centering
    \includesvg[height = 19em]{../pictures/components-of-kubernetes-edit.svg}
\end{frame}

\begin{frame}
    \centering
    \includegraphics[height=18em]{../pictures/k8s-1.30.png} \\
\end{frame}
