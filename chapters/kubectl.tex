\section{Kubectl}
\subsection{Wie steuer ich mein Cluster?}
\begin{frame}
    \slidehead
    \Large
    \centering
    Die Cluster-Verwaltung wird mit dem command-line-tool
    \bashcommand{kubectl}
    gemacht.
\end{frame}

\begin{frame}
    \slidehead
    \vspace{-0.5em}
    \bashcommand{kubectl get nodes}
    gibt die aktuellen nodes aus dem Cluster zurück und deren Status
    \bashcommand{kubectl get pods}
    gibt die aktuell existierenden Container zurück inklusive Status
    \bashcommand{kubectl describe node|pod IDENTIFIER}
    Gibt den kompletten Status von dem jeweiligem Objekt(Node oder Pod) zurück
\end{frame}

\subsection{Deployments deployen}
\begin{frame}
    \slidehead
    \vspace{-0.5em}
    \Large
    \begin{center}
    Wie deploy ich jetzt Dinge auf dem Cluster?
    \normalsize
    
    \bashcommand{kubectl apply -f deployment.yaml}
    \vspace{1em}
    \bashcommand{kubectl delete -f deployment.yaml}
    entfernt das Deployment aus dem Cluster
    \end{center}
\end{frame}

\subsection{Netzwerkoverlay}
\begin{frame}
    \slidehead
    \Large
    \centering
    Die Nodes brauchen ein separates Netzwerkoverlay, um miteinander zu kommunizieren. \\
    Flannel ist eine simple Lösung ohne viel Konfiguration. \\
    Es können auch andere Netzwerkoverlays verwendet werden, die das \textbf{Container Network Interface} unterstützen
    \normalsize
    \bashcommand{kubectl apply -f kube-flannel.yml}
\end{frame}