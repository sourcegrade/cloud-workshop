\section{Dockerfile}\label{sec:dockerfile}
\subsection{Wie aus einem Projekt ein Image wird}

\begin{frame}
    \slidehead
    \Large
    \begin{itemize}[<+->]
        \item Eine \textbf{Dockerfile} wird dazu benutzt, um aus eine Repository ähnlich zu einem Build-Process
              ein Image zu erstellen.
        \item Dazu wird die Datei \textbf{Dockerfile} erstellt, die die Instruktionen enthält, wie das Image gebaut wird.
        \item In dieser sind Iterativ alle Schritte aufgelistet.
    \end{itemize}
\end{frame}

\subsection{Beispiel-Dockerfile}
\begin{frame}[fragile]
    \slidehead
    \centering
    \begin{codeBlock}{title=\codeBlockTitle{Dockerfile}}
        FROM node:18-alpine //Basisimage für Grunddependencies
        WORKDIR /app //alle commands ab hier werden in /app ausgeführt
        COPY . . //nimmt alles aus der Repository mit
        RUN yarn install --production //führt Befehl im Image aus
        CMD ["node", "src/index.js"] //Entrypoint für das Image
        EXPOSE 3000 //interner Port der exposed werden soll
    \end{codeBlock}
\end{frame}

\subsection{Wichtigsten Befehle}

\begin{frame}
    \slidehead
    \Large
    \begin{itemize}[<+->]
        \item \textbf{FROM:} Nimmt ein existierendes Image, in dem Fall node mit der Version 18-alpine,
              und baut auf diesem auf.
        \item \textbf{WORKDIR: } Spezifiziert den Ordner, in der die App gebaut werden soll, damit sie nicht im
              /-Ordner (root) gebaut wird.
        \item \textbf{COPY: } Gibt an, welche Dateien aus der \textbf{Source-Repository} mitgenommen werden sollen.
              Dabei ist das erste Argument aus der Repository während das zweite im Image ist.
    \end{itemize}
\end{frame}

\begin{frame}
    \slidehead
    \Large
    \begin{itemize}[<+->]
        \item \textbf{RUN: } führt einen beliebigen command, der in der Repository verfügbar ist aus.
              Die Ausgabedateien bleiben dabei im Image. Soll dazu benutzt werden das Projekt zu Bauen.
        \item \textbf{CMD: oder ENTRYPOINT:} beim starten des Images wird dieser Befehl ausgeführt.
        \item \textbf{ARG: } setzt beim Bauen des Images die Umgebungsvariable var=wert
    \end{itemize}
\end{frame}

\subsection{Bauen des Images}
\begin{frame}
    \slidehead
    \Large
    \centering
    Ein Image kann ganz einfach mit dem Befehl:
    \vspace{1em}
    \normalsize
    \bashcommand{docker build .}
    \vspace{1em}
    \Large
    in der Repository wo die Dockerfile ist, gebaut werden.
\end{frame}



